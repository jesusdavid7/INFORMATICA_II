\documentclass{article}
\usepackage[utf8]{inputenc}
\usepackage[spanish]{babel}
\usepackage{listings}
\usepackage{graphicx}
\graphicspath{ {images/} }
\usepackage{cite}

\begin{document}

\begin{titlepage}
    \begin{center}
        \vspace*{1cm}
            
        \Huge
        \textbf{Parcial 1 - Calistenia}
            
        \vspace{0.5cm}
        \LARGE
        instrucciones
            
        \vspace{1.5cm}
            
        \textbf{Jesus David Tovar Serrano}
            
        \vfill
            
        \vspace{0.8cm}
            
        \Large
        Despartamento de Ingeniería Electrónica y Telecomunicaciones\\
        Universidad de Antioquia\\
        Medellín\\
        Marzo de 2021
            
    \end{center}
\end{titlepage}

\tableofcontents
\newpage
\section{Descripción del problema}\label{intro}
Esta es la primera sección, podemos agregar algunos elementos adicionales y todo será escrito correctamente. Más aún, si una palabra es demasiado larga y tiene que ser truncada, babel tratará de truncarla correctamente dependiendo del idioma.

\section{Guía de solución} \label{contenido}
Con el objetivo de crear una posible solución al problema he creado una serie de pasos a seguir para que cualquier persona logre llevar las tarjetas de su posición inicial a su posición final.

\begin{enumerate}
  \item Por favor siga las siguientes instrucciones de manera rigurosa:
  \begin{enumerate}
  \item Another entry in the list
  \item Another entry in the list
\end{enumerate}
\end{enumerate}

\section{Inclusión de imágenes} \label{imagenes}

En la Figura (\ref{fig:cpplogo}), se presenta el logo de C++ contenido en la carpeta images.

\begin{figure}[h]
\includegraphics[width=4cm]{cpplogo.png}
\centering
\caption{Logo de C++}
\label{fig:cpplogo}
\end{figure}

Las secciones (\ref{intro}), (\ref{contenido}) y (\ref{imagenes}) dependen del estilo del documento.


\section{Documentación de las pruebas realizadas con 3 personas}
En el siguiente link encontrara un video donde se registro el intento de 3 personas por resolver el problema:

\end{document}
